\documentclass{article}
\usepackage{../../latex_styles/standard-4}

\def\ds{\displaystyle}
\def\pd{\partial}
\def\mb{\mathbf}
\def\mbg{\boldsymbol}
\def\grad{\mathbf\nabla}

\begin{document}
\section{PGD Decomposition}
Consider the 1D transient heat equation with constant and uniform diffusivity, $k$, and source term, $f$:
\begin{equation}
\label{eq::GenHeatEqn}
\frac{\pd u}{\pd t} -k\cdot \grad^2 u = f
\end{equation}

The PGD solution seeks a decomposed form of the solution, $u$. We can decompose Equation \ref{eq::GenHeatEqn} in a number of ways, however, for the purpose of this exercise we will decompose the solution in space, $x$, time, $t$, and diffusivity, $k$. We will define the dimensionality of the problem as $\left( x,t,k \right) \in \Omega$, where $\Omega = \Omega_x \times \Omega_t \times \Omega_k$ with $\Omega_x \subset \mathbb{R}, \Omega_t \subset \mathbb{R},$ and $\Omega_k \subset \mathbb{R}$.
\begin{equation}
u(x,t,k) = \ds\sum_{i=1}^n X_i(x) \cdot T_i(t) \cdot K_i(k)
\end{equation}

\vspace{-6pt}where:\\[-3pt]

\hspace*{15pt}\begin{minipage}{4.5in}

$n$ 
\parbox[t][0.65cm]{4.5in}{ is the total number of ``enrichment steps" in the solution.}

$X_i(x)$ 
\parbox[t][0.65cm]{4.5in}{ $\to$ basis vector defining the spatial dependence at enrichment step, $i$.}

$T_i(t)$ 
\parbox[t][0.65cm]{4.5in}{ $\to$ basis vector defining the time dependence at enrichment step, $i$.}

$K_i(k)$ 
\parbox[t][0.65cm]{4.5in}{ $\to$ basis vector defining the diffusivity dependence at enrichment step, $i$.}

\end{minipage}\vspace{5pt}

We can compute an additional enrichment step in the following manner:
\begin{equation}
\label{eq::TrialFctn}
u_{n+1}(x,t,k) = \ds\sum_{i=1}^n X_i(x) \cdot T_i(t) \cdot K_i(k) + X_{n+1}(x) \cdot T_{n+1}(t) \cdot K_{n+1}(k)
\end{equation}

For notational simplicities sake, we can define $X_{n+1}(x)$, $T_{n+1}(t)$, and $K_{n+1}(k)$ as $R(x)$, $S(t)$, and $W(k)$, respectively.  

We can define a weighted residual form of Equation \ref{eq::GenHeatEqn} with $u^*$ defined as a suitable test function:
\begin{equation}
\label{eq::WeightedResidualForm}
\ds\int_{\Omega} u^* \cdot\left(\frac{\pd u}{\pd t} -k\cdot\grad^2 u -f \right) dx \cdot dt \cdot dk = 0
\end{equation}

Using Equation \ref{eq::TrialFctn} as the definition for $u$, Equation \ref{eq::WeightedResidualForm} yields a non-linear problem that solves for the unknown functions $R(x)$, $S(t)$, and $W(k)$. For this work, Equation \ref{eq::WeightedResidualForm} will be solved iteratively using an alternating direction scheme. This will treat the problem as three successive subproblems: first, solving for $R(x)$ using an assumed solution for $S(t)$ and $W(k)$; second, solving for $S(t)$ using the just calculated $R(x)$ and an assumed solution for $W(k)$; third, solving for $W(k)$ using the just updated $R(x)$ and $S(t)$ values. 

\subsection{Solving for $\mathbf{R(x)}$ from $\mathbf{S(t)}$ and $\mathbf{W(k)}$}
We can define a test function as such:
\begin{equation}
\label{eq::R-TestFctn}
u^*(x,t,k) = R^*(x) \cdot S(t) \cdot W(k)
\end{equation}

Now, plugging Equations \ref{eq::R-TestFctn} and \ref{eq::TrialFctn} into the weighted residual form (Equation \ref{eq::WeightedResidualForm}):
\begin{align*}
\ds\int_{\Omega_x \times \Omega_t \times \Omega_k} & R^*(x) \cdot S(t) \cdot W(k) \bigg[ \frac{\pd}{\pd t} \left( R(x) \cdot S(t) \cdot (W(k) \right) - \\
&  k\grad^2 \cdot \left( R(x) \cdot S(t) \cdot (W(k) \right) -f \bigg] dx \cdot dt \cdot dk = 
\end{align*}
\begin{align}
\label{eq::R-1}
- \ds\int_{\Omega_x \times \Omega_t \times \Omega_k} & R^*(x) \cdot S(t) \cdot W(k) \Bigg[\frac{\pd}{\pd t} \ds\sum_{i=1}^{n-1} \left(X_i(x) \cdot T_i(t) \cdot K_i(k)\right) - \nonumber\\
&  k\grad^2 \cdot \left( \frac{\pd}{\pd t} \ds\sum_{i=1}^{n-1} \left(X_i(x) \cdot T_i(t) \cdot K_i(k)\right) \right)\Bigg] dx \cdot dt \cdot dk 
\end{align}

The RHS of Equation \ref{eq::R-1} is known as the residual and reduces to a known scalar value at time step $n$ (i.e. the previous step in which we know all of the information). We can begin to simplify Equation \ref{eq::R-1} by distributing and collecting terms:
\begin{equation*}
\ds\int_{\Omega_x} R^*(x) \Bigg[ R(x)\cdot \underbrace{\ds\int_{\Omega_t} S(t)\frac{\pd S}{\pd t}dt}_{s_2} \cdot \underbrace{\ds\int_{\Omega_k} W(k)^2 dk}_{w_1} - k\grad^2 R(x) \cdot \underbrace{\ds\int_{\Omega_t} S(t)^2 dt}_{s_1} \cdot \underbrace{\ds\int_{\Omega_k} kW(k)^2 dk}_{w_2} \Bigg] dx = 
\end{equation*}\vspace{-15pt}
\begin{align}
-\ds\int_{\Omega_x}& R^*(x) \Bigg[ \ds\sum_{i=1}^{n-1} \bigg(X_i(x) \cdot \underbrace{\ds\int_{\Omega_t} S(t)\frac{dT_i(t)}{dt} dt}_{s_4} \cdot \underbrace{\ds\int_{\Omega_k}W(k)K_i(k) dk}_{w_4} - \nonumber \\
& \grad^2X_i(x) \cdot \underbrace{\ds\int_{\Omega_t}S(t)T_i(t) dt}_{s_5} \cdot \underbrace{\ds\int_{\Omega_k}kW(k)K_i(k)dk}_{w_5} \bigg) + f \cdot \underbrace{\ds\int_{\Omega_t} S(t) dt}_{s_3} \cdot \underbrace{\ds\int_{\Omega_k} W(k)dk}_{w_3} \Bigg] dx
\end{align}

Rewriting the above equation, we can obtain a weak form that can be solved via a suitable discretization scheme like finite elements:
\begin{gather}
\ds\int_{\Omega_x} R^*(x) \Bigg[ R(x)\cdot s_2 \cdot w_1 - k\grad^2 R(x) \cdot s_1 \cdot w_2 \Bigg] dx = \nonumber \\
 -  \ds\int_{\Omega_x} R^*(x) \Bigg[ \ds\sum_{i=1}^{n-1} \bigg(X_i(x) \cdot s_4 \cdot w_4 - \grad^2X_i(x) \cdot s_5 \cdot w_5 \bigg) + f \cdot s_3 \cdot w_3 \Bigg] dx
\end{gather}

We can also consider the strong form which can be solved numerically through finite difference methods:
\begin{gather}
\label{eq::R-strong}
R(x)\cdot s_2 \cdot w_1 - k\grad^2 R(x) \cdot s_1 \cdot w_2 =  \nonumber \\
\ds\sum_{i=1}^{n-1} \bigg(X_i(x) \cdot s_4 \cdot w_4 - \grad^2X_i(x) \cdot s_5 \cdot w_5 \bigg) + f \cdot s_3 \cdot w_3 
\end{gather}

\subsection{Solving for $\mathbf{S(t)}$ from $\mathbf{R(x)}$ and $\mathbf{W(k)}$}
To solve for $S(t)$ we will define a new test function:
\begin{equation}
\label{eq::R-TestFctn-2}
u^*(x,t,k) = R(x) \cdot S^*(t) \cdot W(k)
\end{equation}

As before in the previous section, we can insert Equations \ref{eq::R-TestFctn-2} and \ref{eq::TrialFctn} into Equation \ref{eq::WeightedResidualForm}, distributing and collecting terms, and obtain the following:
\begin{equation*}
\ds\int_{\Omega_t} S^*(t) \bigg[ \frac{\pd S}{\pd t} \cdot \underbrace{\ds\int_{\Omega_x} R(x)^2 dx}_{r_1} \cdot \underbrace{\ds\int_{\Omega_k}W(k)^2 dk}_{w_1} - S(t) \cdot \underbrace{\ds\int_{\Omega_k}kW(k)^2dk}_{w_2} \cdot \underbrace{\int_{\Omega_x} R(x)\cdot\grad^2R(x) dx}_{r_2} \bigg]
\end{equation*}\vspace{-15pt}
\begin{align}
= & - \ds\int_{\Omega_t} S^*(t) \Bigg[ \ds\sum_{i=1}^{n-1} \bigg( \frac{\pd T_i}{\pd t} \cdot  \underbrace{\ds\int_{\Omega_x} R(x) X_i(x) dx}_{r_5} \cdot \underbrace{\ds\int_{\Omega_k}W(k)K_i(k) dk}_{w_4} \nonumber\\
 - & T_i(t)\cdot \underbrace{\ds\int_{\Omega_k} KW(k)K_i(k)dk}_{w_5} \cdot \underbrace{\ds\int_{\Omega_x} R(x)\cdot\grad^2X_i(x)}_{r_4} \bigg) + f \cdot \underbrace{\ds\int_{\Omega_x} R(x)dx}_{r_3} \cdot \underbrace{\ds\int_{\Omega_t} W(k)dk}_{w_3} \Bigg] dt
\end{align}

Rewriting the above equation, we can obtain a weak form that can be solved via a suitable discretization scheme like finite elements:
\begin{gather}
\ds\int_{\Omega_t} S^*(t) \bigg[ \frac{\pd S}{\pd t} \cdot r_1 \cdot w_1- S(t) \cdot w_2 \cdot r_2 \bigg] \nonumber \\
= - \ds\int_{\Omega_t} S^*(t) \Bigg[ \ds\sum_{i=1}^{n-1} \bigg( \frac{\pd T_i}{\pd t} \cdot r_5 \cdot w_4 - T_i(t)\cdot w_5 \cdot r_4 \bigg) + f \cdot r_3 \cdot w_3 \Bigg] dt
\end{gather}

We can also consider the strong form which can be solved numerically through finite difference methods:
\begin{equation}
\label{eq::S-Strong}
\frac{\pd S}{\pd t} \cdot r_1 \cdot w_1 - S(t) \cdot w_2 \cdot r_2 = \ds\sum_{i=1}^{n-1} \bigg( \frac{\pd T_i}{\pd t} \cdot r_5 \cdot w_4 - T_i(t)\cdot w_5 \cdot r_4 \bigg) + f \cdot r_3 \cdot w_3
\end{equation}

\subsection{Solving for $\mathbf{W(k)}$ from $\mathbf{R(x)}$ and $\mathbf{S(t)}$}
To solve for $W(k)$ we will define a new test function:
\begin{equation}
\label{eq::R-TestFctn-3}
u^*(x,t,k) = R(x) \cdot S(t) \cdot W^*(k)
\end{equation}

As before in the previous sections, we can insert Equations \ref{eq::R-TestFctn-3} and \ref{eq::TrialFctn} into Equation \ref{eq::WeightedResidualForm}, distributing and collecting terms, and obtain the following:
\begin{equation*}
\ds\int_{\Omega_t} W^*(k) \bigg[W(k) \cdot \underbrace{\ds\int_{\Omega_x} R(x)^2 dx}_{r_1} \cdot \underbrace{\ds\int_{\Omega_t} S(t)\frac{\pd S}{\pd t} dt}_{s_2} - kW(k) \cdot \underbrace{\ds\int_{\Omega_t} S(t)^2 dt}_{s_1} \cdot \underbrace{\ds\int_{\Omega_x} R(x)\cdot \grad^2R(x) dx}_{r_2} \bigg] 
\end{equation*}\vspace{-15pt}
\begin{align}
=  -& \ds\int_{\Omega_k} W^*(k) \Bigg[\ds\sum_{i=1}^{n-1} \bigg( K_i(k) \cdot \underbrace{\ds\int_{\Omega_x} R(x)\cdot X_i(x) dx}_{r_5} \cdot \underbrace{\ds\int_{\Omega_t} S(t) \frac{\pd T_i}{\pd t} dt}_{s_4} \nonumber\\
- & K_i k \cdot \underbrace{\ds\int_{\Omega_x} R(x)\cdot \grad^2 X-i(x)dx}_{r_4} \cdot \underbrace{\ds\int_{\Omega_t} S(t)T_i(t)dt}_{s_5}\bigg) + f\cdot \underbrace{\ds\int_{\Omega_x} R(x) dx }_{r_3} \cdot \underbrace{\ds\int_{\Omega_t}S(t)dt}_{s_3} \Bigg]
\end{align}

Simplifying the above equation, we can obtain an analytical solution for the unknown $W(k)$ value:
\begin{equation}
\label{eq::W-analytical}
W(k) =  \frac{-\ds\sum_{i=1}^{n-1} \bigg( K_i(k) \cdot r_5 \cdot s_4 -  K_i k \cdot r_4 \cdot s_5\bigg) + f\cdot r_3 \cdot s_3}{r_1 \cdot s_2 - k \cdot s_1 \cdot r_2}
\end{equation}

\section{Numerical Test Problem}
For this work, we will solve equations \ref{eq::R-strong}, \ref{eq::S-Strong}, and \ref{eq::W-analytical} to approximate the solution of Equation \ref{eq::GenHeatEqn}. The space-time-diffusivity domain is defined as such: $\Omega_x = (0,1), \Omega_t = (0,0.1], \Omega_k = (1,5)$. The unknown functions $X_i(x)$, $T_i(t)$, and $K_i(k)$ are sought on uniform grids with 61 points for the space domain, 151 points for the time domain, and 101 points in the diffusivity domain. 

\subsection{Computing $\mathbf{R(x)}$}
Equation \ref{eq::R-strong} is discretized using second order finite differences (trapezoidal method) for all first and second order derivatives. Each of the integrals are also solved through the well known trapezoidal method for approximating integrals (also second order). Approaching the discretization of left hand side of Equation \ref{eq::R-strong} by a term by term basis we can obtain:
\begin{align*}
-\grad^2R(x) &\to \frac{R(i+1)-2R(i)+R(i-1)}{2\Delta x}     &     \ds\int_{\Omega_t} S(t)^2 dt &\to \frac{\Delta t}{2} \left( S(i)^2 + S(i-1)^2 \right) \\
\ds\int_{\Omega_k} KW(k)^2 dk &\to \frac{k\Delta k}{2}\left( W(i)^2+W(i-1)^2\right)     &    \ds\int_{\Omega_k} W(k)^2 dk &\to  \frac{\Delta k}{2}\left(W(i)^2 + W(i-1)^2 \right) 
\end{align*}
\begin{equation*}
\ds\int_{\Omega_t} S(t)\frac{\pd S}{\pd t} dt \to \frac{1}{4} \left[ S(i)\left( \frac{S(i+1)-S(i-1)}{2} \right) + S(i-1)\left(\frac{S(i)-S(i-2)}{2}\right) \right] 
\end{equation*}

Now for the right hand side:
\begin{equation*}
\ds\int_{\Omega_t} S(t) \frac{dT_j(t)}{dt} dt \to \frac{1}{4} \left[ S(i) \left(\frac{T_j(i+1)-T_j(i-1)}{2} \right) + S(i-1)\left(\frac{T_j(i)-T_j(i-2)}{2} \right) \right]
\end{equation*}
\begin{align*}
\ds\int_{\Omega_k} W(k)K_j(k) dk &\to \frac{\Delta K}{2} \left(W(i)K_j(i) + W(i-1)K_j(i-1) \right)   &   \grad^2X_j(x) &\to \frac{X_i(i+1)-2X_i(i)+X_i(i-1)}{2\Delta x} \\
\ds\int_{\Omega_t} S(t)T_j(t) dt &\to \frac{\Delta T}{2} \left(S(i)T_j(i) + S(i-1)T_j(i-1) \right)   &   \ds\int_{\Omega_k} kW(k)K_j(k) dk &\to \frac{k\Delta K}{2} \left(W(i)K_j(i) + W(i-1)K_j(i-1) \right)
\end{align*}
















\end{document}