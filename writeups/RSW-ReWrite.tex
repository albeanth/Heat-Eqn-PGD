\documentclass{article}
\usepackage{../../latex_styles/standard-4}

\def\ds{\displaystyle}
\def\pd{\partial}
\def\mb{\mathbf}
\def\mbg{\boldsymbol}
\def\grad{\mathbf\nabla}

\begin{document}
\section*{R(x)}
\begin{align}
R(x)\cdot w_1\cdot  s_2 - \grad^2 R(x)\cdot w_2\cdot  s_1 = \ds\sum_{j=1}^{n-1} \left( -w_4^j\cdot s_4^j\cdot X^j(x) + \grad^2 X^j(x)\cdot w_5^j\cdot s_5^j \right) + w_3\cdot s_3\cdot f
\end{align}

Using second order finite differences, we can discretize the above equation as follows:\\
------------------------\\
for $i=(0,61-1)$
\begin{align}
R_i & \cdot w_1\cdot  s_2 - \left( \frac{R_{i+1} - 2R_i + R_{i-1}}{\Delta x^2} \right) \cdot w_2\cdot  s_1 \nonumber\\
= &\ds\sum_{j=1}^{n-1} \left( - X^j_i \cdot w_4^j\cdot s_4^j + \left( \frac{X^j_{i+1} - 2X^j_i + X^j_{i-1}}{\Delta x^2} \right) \cdot w_5^j\cdot s_5^j \right) + w_3\cdot s_3\cdot f
\end{align}
Rewriting the above:
\begin{align}
\label{eq::finalForm}
&-R_{i-1} \frac{s_1 \cdot w_2}{\Delta x^2} + R_i \left( s_2 \cdot w_1 + \frac{2\;s_1 \cdot w_2}{\Delta x^2} \right) - R_{i+1} \frac{ s_1 \cdot w_2}{\Delta x^2} \nonumber \\
= &\ds\sum_{j=1}^{n-1} \left( - X^j_i \cdot w_4^j\cdot s_4^j + \left( \frac{X^j_{i+1} - 2X^j_i + X^j_{i-1}}{\Delta x^2} \right) \cdot w_5^j\cdot s_5^j \right) + w_3\cdot s_3\cdot f
\end{align}
------------------------\\
\vspace{-6pt}We can define the following constants and put Equation \ref{eq::finalForm} into matrix form :\\[-10pt]

\hspace*{15pt}\begin{minipage}{4.5in}
\begin{flalign*}
v_1 &= -\frac{s_1 \cdot w_2}{\Delta x^2}\quad, \qquad v_2 = \frac{ 2\; s_1\cdot w_2}{\Delta x^2}+s_2\cdot w_1\quad, \qquad v_3 = v_1 \\[5pt]
b & = \ds\sum_{j=1}^{n-1} \left( - X^j_i \cdot w_4^j\cdot s_4^j + \left( \frac{X^j_{i+1} - 2X^j_i + X^j_{i-1}}{\Delta x^2} \right) \cdot w_5^j\cdot s_5^j \right) + w_3\cdot s_3\cdot f
\end{flalign*}
\end{minipage}\vspace{5pt}

\begin{equation}
\begin{bmatrix}
1    & 0    &  0        & \dots & 0 \\
v_1    &  v_2    &  v_3    & \dots & 0 \\
\vdots&  \ddots& \ddots & \ddots & \vdots  \\
\vdots&  & v_1  & v_2   &  v_3 & \\
0        & \dots&      0       &  0     & 1 &
\end{bmatrix}
\begin{bmatrix}
R_0 \\ R_1 \\ \vdots \\R_{q-1} \\R_q
\end{bmatrix}
= \begin{bmatrix}
0 \\ b_1 \\ \vdots \\ b_{q-1} \\ 0
\end{bmatrix}
\end{equation}

\newpage
\section*{S(t)}
\begin{align}
w_1 \cdot r_1 \cdot \frac{dS(t)}{dt} - w_2 \cdot r_2 \cdot S(t) = -\ds\sum_{j=1}^{n-1} \left( w_4^j \cdot r_5^j \cdot \frac{dT^j(t)}{dt} - w_5^j \cdot r_4^j \cdot T^j(t) \right) + w_3 \cdot r_3 \cdot f
\end{align}
Using backward euler, we can discretize the above equation as follows:  \\
------------------------\\
for $i=1:(151-1)$ ; $S(i=0,t) = 0$ 
\begin{align}
w_1 \cdot r_1 & \left( \frac{S_{i+1} - S_i}{\Delta t} \right) - w_2 \cdot r_2 \cdot S_{i+1} \nonumber \\
& = -\ds\sum_{j=1}^{n-1} \left( w_4^j \cdot r_5^j \cdot \frac{T^j_{i+1} - T^j_i}{\Delta t} - w_5^j \cdot r_4^j \cdot T^j_{i+1} \right) + w_3 \cdot r_3 \cdot f
\end{align}
------------------------\\
Can be rewritten as:
\begin{align}
S_{i+1} & \left( \frac{w_1 \cdot r_1}{\Delta t} - w_2 \cdot r_2 \right) - \frac{w_1 \cdot r_1}{\Delta t}S_i \nonumber \\ 
& = - \ds\sum_{j=1}^{n-1} \left( w_4^j \cdot r_5^j \cdot \frac{T^j_{i+1} - T^j_i}{\Delta t} - w_5^j \cdot r_4^j \cdot T^j_{i+1} \right) + w_3 \cdot r_3 \cdot f
\end{align}

And finally discretized in the following form:
\begin{align}
S_{i+1} =  \frac{\frac{w_1 \cdot r_1}{\Delta t}S_i - \ds\sum_{j=1}^{n-1} \left( w_4^j \cdot r_5^j \cdot \frac{T^j_{i+1} - T^j_i}{\Delta t} - w_5^j \cdot r_4^j \cdot T^j_{i+1} \right) + w_3 \cdot r_3 \cdot f}{\left( \frac{w_1 \cdot r_1}{\Delta t} - w_2 \cdot r_2 \right)}
\end{align}

\section*{W(k)}
Solving for W(k) is of an analytical form and can be represented as follows:  
\begin{align}
W(k) = \frac{ -\ds\sum_{j=1}^{n-1} \left( r_5^j \cdot s_4^j \cdot K^j(k) - r_4^j \cdot s_5^j \cdot k \cdot K^j(k) \right) + r_3 \cdot s_3 \cdot f }{r_1 \cdot s_2 - r_2 \cdot s_1 \cdot k}
\end{align}

\newpage
\section*{Integral Evaluations}
All of the following integrals are evaluated with the trapezoidal rule. Second order derivatives are approximated using second order finite differences. First order derivatives are approximated using backward differencing. 
\begin{align*}
\intertext{for i=(0,60-1)}
r_1 &= \ds\int_{\Omega_x} R(x)^2 dx \: \approx \: \frac{\Delta x}{2} \left( R_{i+1}^2 + R_i^2 \right) \\[5pt]
r_2 &= \int_{\Omega_x} R(x)\cdot\grad^2R(x) dx \: \approx \: \frac{1}{2\Delta x} \left( R_{i+1} \left( R_{i+2}-2R_{i+1}+R_{i} \right) + R_{i} \left( R_{i+1}-2R_{i}+R_{i-1} \right) \right)^*\\[5pt]
r_3 &= \ds\int_{\Omega_x} R(x)dx \: \approx \: \frac{\Delta x}{2} (R_{i+1} + R_i) \\[5pt]
r^j_4 &= \ds\int_{\Omega_x} R(x)\cdot\grad^2X^j(x) \: \approx \: \frac{1}{2\Delta x} \left( R_{i+1} \left( X^j_{i+2}-2X^j_{i+1}+X^j_{i}\right) + R_i \left(X^j_{i+1}-2X^j_i+X^j_{i-1}\right) \right)^*\\
r^j_5 &= \ds\int_{\Omega_x} R(x) X^j(x) dx \: \approx \: \frac{\Delta x}{2} \left( R_{i+1}X^j_{i+1} + R_iX^j_i \right) \\[5pt]
        &\text{\small{\emph{$\quad\:\:^*$ if $i=0$, $R_i=X^j_i=0$ or $i=59$, $R_{i+1}=X^j_{i+1}=0$}}} \\[25pt]
\intertext{for i=(0,150-1)}
s_1 &= \ds\int_{\Omega_t} S(t)^2 dt \: \approx \: \frac{\Delta t}{2} \left( S_{i+1}^2 + S_i^2 \right) \\[5pt]
s_2 &= \ds\int_{\Omega_t} S(t)\frac{dS}{dt}dt \: \approx \: \frac{1}{2} \left( S_{i+1}\left( S_{i+1}-S_i \right) + S_i\left(S_i-S_{i-1}\right) \right) \\
s_3 &= \ds\int_{\Omega_t} S(t) dt \: \approx \: \frac{\Delta t}{2} \left( S_{i+1} + S_i \right) \\[5pt]
s^j_4 &= \ds\int_{\Omega_t} S(t)\frac{dT^j(t)}{dt} dt \: \approx \: \frac{1}{2} \left( S_{i+1} \left(T^j_{i+1}-T^j_i \right) + S_i\left(T^j_i-T^j_{i-1}\right) \right) \\[5pt]
s^j_5 &= \ds\int_{\Omega_t}S(t)T^j(t) dt \: \approx \: \frac{\Delta t}{2} \left(S_{i+1}T^j_{i+1} + S_iT^j_i \right) \\[5pt]
        &\text{\small{\emph{$\quad\:\:^*$ if $i=0$, $S_i=T^j_i=0$}}} \\[25pt]
\intertext{for i=(0,100-1)}
w_1 & = \ds\int_{\Omega_k} W(k)^2 dk \: \approx \: \frac{\Delta k}{2}\left(W_{i+1}^2 + W_i^2 \right)  \\[5pt]
w_2 & = \ds\int_{\Omega_k} kW(k)^2 dk \: \approx \: \frac{\Delta k}{2}\left( k_{i+1}W_{i+1}^2+k_iW_i^2\right)   \\[5pt]
w_3 & = \ds\int_{\Omega_k} W(k)dk \: \approx \: \frac{\Delta k}{2} \left( W_{i+1} + W_i \right) \\[5pt]
w^j_4 & = \ds\int_{\Omega_k}W(k)K^j(k) dk \: \approx \: \frac{\Delta k}{2} \left(W_{i+1}K^j_{i+1} + W_iK^j_i \right) \\[5pt]
w^j_5 & = \ds\int_{\Omega_k}kW(k)K^j(k)dk \: \approx \: \frac{\Delta k}{2} \left(k_{i+1}W_{i+1}K^j_{i+1} + k_iW_iK^j_i \right) \\[5pt]
\end{align*}



\end{document}